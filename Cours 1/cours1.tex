% !TEX encoding = UTF-8 Unicode
\documentclass{tufte-handout}
\usepackage{matlab-prettifier}
\lstset{style=Matlab-editor,frame=single,numbers=left,escapechar=¬}
\hypersetup{colorlinks=true,urlcolor=blue}
\title{Fondements de programmation}
\author{Hugo Leblanc}

%Fucking bullshit bug
%https://tex.stackexchange.com/questions/200722/xetex-seems-to-break-headers-in-tufte-handout
\ifxetex
    \newcommand{\textls}[2][5]{%
    \begingroup\addfontfeatures{LetterSpace=#1}#2\endgroup
    }
    \renewcommand{\allcapsspacing}[1]{\textls[15]{#1}}
    \renewcommand{\smallcapsspacing}[1]{\textls[10]{#1}}
    \renewcommand{\allcaps}[1]{\textls[15]{\MakeTextUppercase{#1}}}
    \renewcommand{\smallcaps}[1]{\smallcapsspacing{\scshape\MakeTextLowercase{#1}}}
    \renewcommand{\textsc}[1]{\smallcapsspacing{\textsmallcaps{#1}}}
\fi

\begin{document}
\maketitle

\section{Environnement}
\subsection{Current Folder}
Le dossier courant\sidenote{\href{https://www.mathworks.com/help/matlab/learn_matlab/desktop.html}{Desktop Basics}} représente l'espace de travail utilisé pour l'utilisation de scripts ou fonctions. Pour pouvoir utiliser les programmes que nous allons créer, il faut que ceux-ci soient dans le dossier courant. On peut modifier le dossier courant avec la barre d'exploration au-dessus de la fenêtre.
\subsection{Command Window}
La fenêtre de commande est l'interface principale dans MATLAB. Elle permet de recevoir les instructions\sidenote{\href{https://www.mathworks.com/help/matlab/matlab_env/enter-statements-in-command-window.html}{Enter Statements in Command Window}} qui seront ensuite exécutées.
\subsection{Workspace}
L'espace de travail représente la mémoire de MATLAB. C'est la représentation des différentes variables qui sont disponibles pour l'exécution.
\subsection{Editor}
L'éditeur est l'endroit où nous travaillerons le plus. C'est ici que nous pouvons écrire nos scripts et fonctions qui seront la base de nos programmes. Il est à mentionner que l'éditeur utilise une barre d'outils avec plusieurs fonctionnalités propres à l'éditeur.

\section{Concepts fondamentaux de MATLAB}
\subsection{Instructions}
Une instruction\sidenote{“command” en anglais.} est l'élément de base que MATLAB traite dans un programme. Une instruction peut exécuter une expression mathématique, appeler une fonction spécifique ou prendre des décisions par rapport à un état ou une condition.
Un programme est constitué d'une multitude d'instructions qui sont exécutées à la chaine par MATLAB. MATLAB est seulement capable d'exécuter une instruction à la fois et doit attendre de compléter l'instruction courante avant de pouvoir en exécuter une autre.
Deux méthodes de bases nous permettent d'utiliser des instructions:
La première et la plus directe est d'utiliser la fenêtre de commande ou "Command Window"\sidenote{L'invite de commande est représentée par deux signes plus grand que ( $>$$>$ ). Cela indique que MATLAB est prêt à recevoir une nouvelle instruction.}. Pour exécuter une instruction, on écrit l'instruction après l'invite de commande suivit de la touche [Entrer] pour exécuter l'instruction.
La deuxième méthode est d'écrire un script. Un script est une série d'instructions écrite d'avance qui sera exécutée de manière séquentielle par MATLAB. Le principe est semblable à copier-coller chaque instruction du script dans la fenêtre de commande puis de l'exécuter manuellement. Les scripts permettent de faire des programmes complexes de manière plus automatique que l'entrée manuelle d'instructions.
Les instructions ont la possibilité de retourner une réponse qui peut être assignée à une variable. Si l'instruction retourne une réponse et qu'aucune variable n'est assignée, la réponse est assignée à la variable "ans" par défaut.
MATLAB affiche le résultat de l'instruction dans l'invite de commande après son exécution.

\subsection{Le point-virgule}
Le point-virgule ( ; )\sidenote{\href{https://www.mathworks.com/help/matlab/matlab_prog/matlab-operators-and-special-characters.html}{MATLAB Operators and Special Characters}} peut être ajouté après une instruction pour ignorer l'affichage du résultat dans la fenêtre de commande. L'instruction est quand même exécutée, mais l'affichage n'est pas présent.
\subsection{Les points de suspension}
Les points de suspension ( ... )\sidenote{\href{https://www.mathworks.com/help/matlab/matlab_prog/continue-long-statements-on-multiple-lines.html}{Continue Long Statements on Multiple Lines}} permettent de faire une instruction sur plusieurs lignes à la place d'une seule. Utilisés dans la fenêtre de commande, MATLAB attend le reste de l'instruction après avoir appuyé sur [Entrer]. Les points de suspension peuvent être utilisés à répétitions sans problèmes.
\subsection{Expressions}
Une expression\sidenote{\href{https://www.mathworks.com/help/matlab/learn_matlab/expressions.html}{Expressions}} est un concept important, car le principe d'expression revient durant tous les concepts de MATLAB.
Dans sa définition la plus simple, une expression est une opération qui donne un résultat. Il existe plusieurs types d'expression, la plus commune est l'expression mathématique.
MATLAB va résoudre les expressions d'une instruction avant de l'exécuter.
\sidenote{On dit qu'une expression est évaluée quand elle est résolue par MATLAB.
Nous utiliserons souvent le terme expression comme substitut dans nos exemples, il faut donc comprendre que n'importe quelle expression peut être insérée à cet endroit.}
La résolution de l'expression est régie par les règles de priorité des opérations\sidenote{\href{https://www.mathworks.com/help/matlab/matlab_prog/operator-precedence.html}{Operator Precedence}}.
\subsection{Fonctions}
MATLAB supporte plusieurs milliers de fonctions\sidenote{\href{https://www.mathworks.com/help/matlab/learn_matlab/calling-functions.html}{Calling Functions}} qui sont mises à notre disposition pour faciliter la conception de nos programmes. Chaque fonction comporte un nom propre et une utilité unique.
\sidenote{”L'appel d'une fonction” est le terme utilisé pour dire qu'on exécute, utilise ou invoque une fonction.}
Les paramètres d'entrées sont des informations primordiales à l'exécution de certaines fonctions, par exemple : l'angle pour calculer un sinus.
On appelle une fonction en écrivant son nom dans une instruction. Si la fonction a besoin de paramètres d'entrées, on utilise des parenthèses pour encapsuler les paramètres et des virgules pour les séparer.

\section{Identificateurs}
Quand vient le temps de créer des éléments nommés dans MATLAB, des règles\sidenote{\href{https://www.mathworks.com/help/matlab/matlab_prog/variable-names.html}{Variable Names}} doivent être suivies. On appelle tout élément nommé de notre propre cru des identificateurs. Les identificateurs sont surtout utilisés pour nommer des fichiers et des variables.
\sidenote{Le tiret bas ( \_ ) est aussi communément appelé "underscore".}
Leur nom doit débuter par une lettre, suivi de lettre, de chiffres ou de tirets bas. MATLAB est sensible à la casse\sidenote{\href{https://www.mathworks.com/help/matlab/matlab_prog/case-and-space-sensitivity.html}{Case and Space Sensitivity}}, il faut donc aussi faire attention à notre utilisation de lettres majuscules ou minuscules dans la création de nos identificateurs.
\sidenote{Par exemple, les fonctions max et min existent déjà, ce qui pourrait être problématique.}
Il faut aussi faire attention de ne pas utiliser des noms de fonctions déjà existantes ou des mots réservés\sidenote{\href{https://www.mathworks.com/help/matlab/ref/iskeyword.html}{iskeyword}}, car la priorité est donnée au nom de variable et la fonction devient donc inaccessible.

\section{Script}
Quand vient le temps de créer des programmes, on veut automatiser l'exécution des instructions. Les scripts\sidenote{\href{https://www.mathworks.com/help/matlab/matlab_prog/create-scripts.html}{Create Scripts}} de MATLAB permettent de faire l'exécution automatique d'instructions écrites dans le fichier.
Les scripts sont des fichiers textes avec une extension .m . Chaque ligne du fichier représente une instruction que MATLAB devra exécuter. Les instructions seront exécutées de manières séquentielles à partir du début du fichier, c'est à dire en ordre de lecture.
N'importe quelle instruction utilisée dans la fenêtre de commande peut être incluse de la même manière dans un script. Le script peut aussi contenir des lignes vides pour augmenter la lisibilité.
Pour exécuter le script, on appuie sur le bouton [Run] de MATAB quand le script est sélectionné dans l'éditeur. On peut aussi exécuter le script en appelant le nom du script dans la fenêtre de commande sans son extension.
\subsection{Commentaires}
Les commentaires\sidenote{\href{https://www.mathworks.com/help/matlab/matlab_prog/comments.html}{Add Comments to Programs}} sont des notes laissés par le programmeur qui ne sera pas exécuté par MATLAB. Les commentaires permettent de laisser des explications utiles sur les instructions dans nos programmes.
Les commentaires commencent avec un symbole de pourcentage ( \% ) et le restant de la ligne est automatiquement en commentaire. Une ligne suivant les points de suspension terminant un commentaire est elle aussi considérée comme tel.
\subsection{Arrêt d'urgence d'un programme}
Si MATLAB ne répond plus ou est constamment "Busy", on peut interrompre\sidenote{\href{https://www.mathworks.com/help/matlab/matlab_env/stop-execution.html}{Stop Execution}} le programme courant avec le raccourci [ Ctrl ] [ C ] après avoir sélectionné la fenêtre de commande. Plusieurs annulations peuvent être nécessaires si plusieurs opérations étaient en attente.
\subsection{Fonctions de bases}
Plusieurs fonctions simples sont utiles en commençant l'utilisation de MATLAB.
\begin{itemize}
    \item clc\sidenote{\href{https://www.mathworks.com/help/matlab/ref/clc.html}{clc}} - Vide la fenêtre de commande
    \item clear\sidenote{\href{https://www.mathworks.com/help/matlab/ref/clear.html}{clear}} - Vide l'espace de travail
    \item help\sidenote{\href{https://www.mathworks.com/help/matlab/ref/help.html}{help}} fonction - Donne de l'aide sommaire sur la fonction
    \item doc\sidenote{\href{https://www.mathworks.com/help/matlab/ref/doc.html}{doc}} fonction - Donne la documentation sur la fonction
\end{itemize}

\section{Variables}
Une variable est l'unité de mémoire de base de nos programmes. Une variable consiste en 3 choses :
\begin{itemize}
    \item Un identificateur - Le nom de la variable
    \item Une valeur - Le contenu de la variable
    \item Un type - La nature de la variable
\end{itemize}
On crée une variable avec une instruction d'assignation. L'assignation utilise le sigle égal ( = ) pour indiquer l'assignation d'une expression à un identificateur.
\sidenote{L'assignation se fait toujours d'une expression à droite vers un identificateur à gauche.}
Pour utiliser une variable déjà assignée, on place son identificateur à l'intérieur d'une expression. La valeur contenue dans la variable sera remplacée dans l'expression avant qu'elle ne soit résolue.
\subsection{Type de données}
Les variables de MATLAB peuvent avoir plusieurs formes\sidenote{\href{https://www.mathworks.com/help/matlab/matlab_prog/fundamental-matlab-classes.html}{Fundamental MATLAB Classes}}. Nous nous concentrerons sur trois types de variables durant le début du cours.
\begin{itemize}
    \item Valeurs numériques - double
    \item Valeurs logiques - logical
    \item Valeurs textuelles - char \sidenote{Les valeurs textuelles sont appelées chaine de caractères.}
\end{itemize}
\subsection{double}
Pour les variables numériques, le type double\sidenote{\href{https://www.mathworks.com/help/matlab/ref/double.html}{double}} est utilisé par défaut dans MATLAB. Ce type est capable de contenir des nombres réels de tout genre.
Il existe une problématique avec le type numérique double. Dû à la gestion de la mémoire pour représenter les valeurs\sidenote{\href{https://www.mathworks.com/help/matlab/matlab_prog/floating-point-numbers.html}{Floating-Point Numbers}}, il est souvent difficile de faire des comparaisons entre deux valeurs qui sont supposées être identiques\sidenote{Exemple : Essayez de faire 0.3 - 0.2 - 0.1 dans la fenêtre de commande.}. Pour faire des comparaisons de valeurs calculées, on veut utiliser une comparaison entre une valeur minimale (ex: 0.01) et la différence absolue des deux valeurs.
\subsection{logical}
Pour des éléments logiques binaires, le type logical\sidenote{\href{https://www.mathworks.com/help/matlab/ref/logical.html}{logical}} est utilisé. Ce type peut avoir seulement deux possibilités: true\sidenote{\href{https://www.mathworks.com/help/matlab/ref/true.html}{true}} ou false\sidenote{\href{https://www.mathworks.com/help/matlab/ref/false.html}{false}}, représentées aussi par les valeurs 1 et 0 respectivement.
\subsection{char}
Pour sauvegarder du texte, le type char\sidenote{\href{https://www.mathworks.com/help/matlab/ref/char.html}{char}} est disponible. Les variables sont capables de contenir des chaines de caractères.
\sidenote{Des chaines de caractères sont nommées "string" en anglais. Il faut faire attention par contre, car il existe un type string dans MATLAB, mais celui-ci ne sera pas utilisé dans le cours, nous utiliserons des “character vectors” de type char.}
Les chaines des caractères sont représentées par du texte à l'intérieur de guillemets simples.
\subsection{Constantes}
Durant la création de nos programmes, il nous est donné d'utiliser des valeurs arbitraires pour représenter des informations (taux de taxation, âge minimum, etc.) Pour augmenter la lisibilité du code et éviter des erreurs, ce genre de valeurs doit être contenu dans des variables avec des identificateurs spéciaux. Les constantes auront des noms de variables constitués seulement de lettres en majuscules pour les différences des variables typiques.
Si notre programme à besoin de plusieurs variables, il est habituel de les regrouper dans un script et d'appeler le script au moment opportun.

\section{fprintf}
La fonction fprintf\sidenote{\href{https://www.mathworks.com/help/matlab/ref/fprintf.html}{fprintf}} permet d'afficher de l'information dans la fenêtre de commande. Son utilisation de base est simple, mais elle contient plusieurs configurations optionnelles.
La version la plus simple du fprintf affiche le contenu d'une chaine de caractères envoyé en paramètre.
\begin{lstlisting}[title={Simple fprintf}]
fprintf('Bonjour la vie!\n')
\end{lstlisting}
Pour afficher le contenu d'une variable avec fprintf, il nous faut inclure un opérateur de formatage à l'intérieur de notre chaine de caractère. Il faut ensuite inclure la valeur à afficher comme expression dans la fonction.
\begin{lstlisting}[title={fprintf avec opérateur de formatage}]
fprintf('Nombre : %g', 3)
fprintf('Mot : %s', 'allo')

% Limite le nombre de décimales avec %.Xf
fprintf('Nombre : %.2f', 3.12345) % Affiche 3.12

x = 5;
fprintf('Nombre : %g', x)
\end{lstlisting}
L'opérateur de formatage utilisé dépend de la valeur à afficher. Les deux cas généraux sont \%g pour des valeurs numériques et \%s pour des chaines de caractères. Il nous est aussi possible d'afficher un nombre limité de valeurs fractionnaires avec \%.Xf ou X est le nombre de décimales à afficher.
La fonction fprintf nous donne aussi la possibilité d'afficher des caractères spéciaux qui ne sont pas représentés par des caractères normaux ou des caractères réservés. Voici les exemples les plus communs.
\begin{itemize}
    \item \textbackslash{}n - Un saut de ligne
    \item \textquotesingle\textquotesingle - Un guillemet simple
    \item \%\% - Un sigle de pourcentage
\end{itemize}

N'oubliez pas qu'on peut utiliser une expression avec l'opérateur de formatage. Cela nous laisse facilement afficher le contenu de variables.

\section{input}
La fonction input\sidenote{\href{https://www.mathworks.com/help/matlab/ref/input.html}{input}} permet de saisir des informations d'un utilisateur. Son appel bloque temporairement l'exécution des autres instructions et MATLAB attend que l'utilisateur entre une information dans la fenêtre de commande.
La fonction s'utilise habituellement avec une assignation et un message de saisie.
\sidenote{Attention, input est incapable d'utiliser des opérateurs de formatage comme fprintf.}
Une autre version est aussi disponible si l'on veut avoir une saisie d'une chaine de caractères.
\begin{lstlisting}[title={Exemples d'utilisations de la fonction input}]
x = input('Entrez un nombre : ')
saisi = input('Entrez un mot : ', 's')
\end{lstlisting}

\section{Opérateurs et caractères spéciaux}
MATLAB est bourré d'opérateurs et de caractères spéciaux\sidenote{\href{https://www.mathworks.com/help/matlab/matlab_prog/matlab-operators-and-special-characters.html}{MATLAB Operators and Special Characters}} à utiliser avec nos instructions. Les opérateurs arithmétiques de bases sont tous présents, mais il existe aussi d'autres types d'opérateurs qui peuvent être utilisés pour générer des expressions.
\subsection{Opérateur relationnels}
Les opérateurs relationnels\sidenote{\href{https://www.mathworks.com/help/matlab/matlab_prog/array-comparison-with-relational-operators.html}{Array Comparison with Relational Operators}} permettent de comparer deux valeurs et donner une réponse logique (true ou false). Les opérateurs relationnels sont :
\begin{itemize}
    \item $<$ - Plus petit que
    \item $<=$ - Plus petit ou égale que
    \item $>$ - Plus grand que
    \item $>=$ - Plus grand ou égale que
    \item $==$ - Égale à \sidenote{L'égalité utilise deux sigles égaux, car le sigle égal unique est réservé pour l'assignation.}
    \item $\ensuremath{\sim}=$ - Différent de
\end{itemize}
Les opérations relationnelles peuvent être jumelées à des opérations arithmétiques qui se feront en premiers d'après la priorité des opérations de MATLAB.

\section{Structure de contrôles}
Les structures de contrôles\sidenote{\href{https://www.mathworks.com/help/matlab/control-flow.html}{Control Flow}} permettent de briser le flot habituel d'exécution des instructions. Les premières structures de contrôles que nous allons voir sont les structures de contrôles conditionnelles, mais il existe aussi des structures de contrôles à boucles.
\subsection{Conditionnel if}
La structure de contrôle conditionnel if\sidenote{\href{https://www.mathworks.com/help/matlab/ref/if.html}{if, elseif, else}} permet de prendre une décision par rapport à une condition et d'exécuter seulement les instructions voulues selon celle-ci.

Il existe plusieurs variations du if. Commençons avec ça forme la plus simple.

\begin{fullwidth}
    \begin{minipage}{.45\linewidth}
    \begin{lstlisting}[title={Structure du if simple}]
if ¬\mlplaceholder{condition}¬
    ¬\mlplaceholder{instructions}¬
end
    \end{lstlisting}
    \end{minipage}\hfill
    \begin{minipage}{.45\linewidth}
    \begin{lstlisting}[title={Exemple du if simple}]
if x > 5
    fprintf('x plus grand que 5.')
end
    \end{lstlisting}
    \end{minipage}
\end{fullwidth}

Dans ce cas simple, la condition, une expression booléenne, est évaluée. Si la condition est vrai, le bloc d'instruction entre le if et le end est exécuté. Sinon, dans le cas où l'expression est fausse, le bloc est ignoré et l'exécution continue après le end.
\sidenote[][-1cm]{Une expression booléenne ou expression logique est une expression qui est évaluée à une valeur logique (true ou false).}
Le deuxième cas ajoute un bloc d'instruction pour une condition fausse.
\begin{fullwidth}
    \begin{minipage}{.45\linewidth}
    \begin{lstlisting}[title={Structure du if avec else}]
if ¬\mlplaceholder{condition}¬
    ¬\mlplaceholder{instructions}¬
else
    ¬\mlplaceholder{instructions}¬
end
    \end{lstlisting}
    \end{minipage}\hfill
    \begin{minipage}{.45\linewidth}
    \begin{lstlisting}[title={Exemple du if avec else}]
if x > 5
    fprintf('x plus grand que 5.')
else
    fprintf('Pas plus que 5.')
end
    \end{lstlisting}
    \end{minipage}
\end{fullwidth}
Avec cette configuration, si la condition est vraie, on exécute le premier bloc d'instruction. Si la condition est fausse, on exécute le bloc entre le else et le end. Dans les deux cas, après avoir exécuté le bloc, on continue l'exécution après le end.
Le troisième cas permet d'avoir des conditions supplémentaires si la première n'est pas vraie.
\begin{fullwidth}
    \begin{minipage}{.45\linewidth}
    \begin{lstlisting}[title={Structure du if avec elseif}]
if ¬\mlplaceholder{condition}¬
    ¬\mlplaceholder{instructions}¬
elseif ¬\mlplaceholder{condition}¬
    ¬\mlplaceholder{instructions}¬
end
    \end{lstlisting}
    \end{minipage}\hfill
    \begin{minipage}{.45\linewidth}
    \begin{lstlisting}[title={Exemple du if avec elseif}]
if x > 5
    fprintf('x plus grand que 5.')
elseif x == 3
    fprintf('x est égale à 3.')
end
    \end{lstlisting}
    \end{minipage}
\end{fullwidth}
Le principe initial reste pareil, si la première condition est vraie, on exécute le premier bloc d'instruction. Sinon, on regarde la deuxième condition et on exécute son bloc d'instructions si la condition est vrai.
Les blocs d'instructions sont mutuellement exclusifs, dès qu'un des blocs est exécuté\sidenote{Les différentes conditions sont évaluées de manières séquentielles. Pour aller voir la deuxième condition, il faut que la première condition soit fausse.}
, le programme saute au end et continu à partir de là.

On peut aussi mélanger les deux configurations, la seule restriction est que le else doit obligatoirement être le dernier bloc d'instructions. Ce dernier sera exécuté seulement si aucune des conditions n'est vraie.
\sidenote{On peut aussi imbriquer les structures de contrôles au besoin. On peut donc avoir des if dans des if.}

\subsection{Conditionnel switch}
La conditionnel switch\sidenote{\href{https://www.mathworks.com/help/matlab/ref/switch.html}{switch, case, otherwise}} est un cas particulier du if avec une syntaxe qui permet de faciliter son implémentation et sa lisibilité.
Le switch est utilisé quand une variable doit être comparée à une série d'égalité possible. On switch sur la variable et implémente plusieurs cas possibles.
\begin{fullwidth}
    \begin{minipage}{.45\linewidth}
    \begin{lstlisting}[title={Structure d'un if pouvant être un switch}]
if x == 3
    ¬\mlplaceholder{instructions}¬
elseif x == 7
    ¬\mlplaceholder{instructions}¬
else
    ¬\mlplaceholder{instructions}¬
end
    \end{lstlisting}
    \end{minipage}\hfill
    \begin{minipage}{.45\linewidth}
    \begin{lstlisting}[title={Structure du switch}]
switch x
    case 3
        ¬\mlplaceholder{instructions}¬
    case 7
        ¬\mlplaceholder{instructions}¬
    otherwise
        ¬\mlplaceholder{instructions}¬
end
    \end{lstlisting}
    \end{minipage}
\end{fullwidth}
Les principes des blocs d'instructions du if sont aussi applicables au switch. Les blocs sont mutuellement exclusifs et il existe un bloc pour le cas où toutes les possibilités ont échoué.
\sidenote{Le switch peut être remplacé par un if dans tous les cas, mais il est préférable de l'utiliser lorsque possible.}
Le switch peut être utilisé pour la comparaison de valeurs numériques ou de chaines de caractères.

\nobibliography{cours1}
\bibliographystyle{plainnat}
\end{document}
