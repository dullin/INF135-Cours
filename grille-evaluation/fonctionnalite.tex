La fonctionnalité est le critère d'évaluation fondamentale. Un programme doit respecter les spécifications du problème donné et fonctionner correctement. Cela inclut de se comporter comme voulu, de produire les bons résultats et sorties pour une variété d'entrées possibles. Cela inclut aussi d'écrire un programme d'une certaine manière ou sous certaines restrictions si cela est demandé dans l'énoncé du travail.

Si la spécification du problème est ambiguë, vous avez deux choix : faire une supposition à propos de ce qui est requis ou demander à l'enseignant. Si vous faites une supposition, vous devriez mentionner dans vos commentaires la supposition utilisée pour que le correcteur soit en connaissance de cause. Une mauvaise supposition apportera une perte de points.
