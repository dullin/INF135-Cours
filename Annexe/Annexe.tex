\documentclass[12pt,landscape]{article}
    \usepackage{multicol}
    %\usepackage{calc}
    \usepackage{ifthen}
    \usepackage[landscape]{geometry}
    %\usepackage{hyperref}
    \usepackage{matlab-prettifier}
    %\usepackage[T1]{fontenc}
    \usepackage{xcolor,beramono,amsmath}
    \usepackage{tikz}
    % To make this come out properly in landscape mode, do one of the following
    % 1.
    %  pdflatex latexsheet.tex
    %
    % 2.
    %  latex latexsheet.tex
    %  dvips -P pdf  -t landscape latexsheet.dvi
    %  ps2pdf latexsheet.ps
    
    
    % If you're reading this, be prepared for confusion.  Making this was
    % a learning experience for me, and it shows.  Much of the placement
    % was hacked in; if you make it better, let me know...
    
    
    % 2008-04
    % Changed page margin code to use the geometry package. Also added code for
    % conditional page margins, depending on paper size. Thanks to Uwe Ziegenhagen
    % for the suggestions.
    
    % 2006-08
    % Made changes based on suggestions from Gene Cooperman. <gene at ccs.neu.edu>
    
    
    % To Do:
    % \listoffigures \listoftables
    % \setcounter{secnumdepth}{0}
    
    
    % This sets page margins to .5 inch if using letter paper, and to 1cm
    % if using A4 paper. (This probably isn't strictly necessary.)
    % If using another size paper, use default 1cm margins.
    \ifthenelse{\lengthtest { \paperwidth = 11in}}
        { \geometry{top=.5in,left=.5in,right=.5in,bottom=.5in} }
        {\ifthenelse{ \lengthtest{ \paperwidth = 297mm}}
            {\geometry{top=1cm,left=1cm,right=1cm,bottom=1cm} }
            {\geometry{top=1cm,left=1cm,right=1cm,bottom=1cm} }
        }
    
    % Turn off header and footer
    \pagestyle{empty}
    
    
    % Redefine section commands to use less space
    \makeatletter
    \renewcommand{\section}{\@startsection{section}{1}{0mm}%
                                    {-1ex plus -.5ex minus -.2ex}%
                                    {0.5ex plus .2ex}%x
                                    {\normalfont\large\bfseries}}
    \renewcommand{\subsection}{\@startsection{subsection}{2}{0mm}%
                                    {-1explus -.5ex minus -.2ex}%
                                    {0.5ex plus .2ex}%
                                    {\normalfont\normalsize\bfseries}}
    \renewcommand{\subsubsection}{\@startsection{subsubsection}{3}{0mm}%
                                    {-1ex plus -.5ex minus -.2ex}%
                                    {1ex plus .2ex}%
                                    {\normalfont\small\bfseries}}
    \makeatother
    
    % Define BibTeX command
    \def\BibTeX{{\rm B\kern-.05em{\sc i\kern-.025em b}\kern-.08em
        T\kern-.1667em\lower.7ex\hbox{E}\kern-.125emX}}
    
    % Don't print section numbers
    \setcounter{secnumdepth}{0}
    
    
    \setlength{\parindent}{0pt}
    \setlength{\parskip}{0pt plus 0.5ex}
    
    
    
    %CURLY BRACES ABOVE
    %https://tex.stackexchange.com/questions/107903/overbrace-in-lstlisting-of-scala-code/107936#107936
    \makeatletter
    \newenvironment{btHighlight}[1][]
    {\begingroup\def\bt@Highlight@par{#1}\begin{lrbox}{\@tempboxa}}
    {\end{lrbox}\bt@HL@box[\bt@Highlight@par]{\@tempboxa}\endgroup}
    
    \newcommand\btHL[1][]{%
    \begin{btHighlight}[#1]\bgroup\aftergroup\bt@HL@endenv%
    }
    \def\bt@HL@endenv{%
    \end{btHighlight}%   
    \egroup
    }
    \newcommand{\bt@HL@box}[2][]{%
      $\overset{\text{#1}}{\overbrace{\strut\usebox{#2}}}$%
    }
    \makeatother
    
    %CURLY BRACES NEXT TO CODE
    \usetikzlibrary{decorations.pathreplacing,calc}
    \newcommand{\tikzmark}[1]{\tikz[overlay,remember picture] \node (#1) {};}
    
    \newcommand*{\AddNote}[4]{%
        \begin{tikzpicture}[overlay, remember picture]
            \draw [decoration={brace,amplitude=0.5em},decorate,ultra thick]
                ($(#3)!([yshift=1.5ex]#1)!($(#3)-(0,1)$)$) --  
                ($(#3)!(#2)!($(#3)-(0,1)$)$)
                    node [align=center, text width=2.5cm, pos=0.5, anchor=west] {#4};
        \end{tikzpicture}
    }%
    
    
    %MATLAB inline mini macro
    \lstMakeShortInline¬
    \lstset{style=Matlab-bw,basicstyle=\mlttfamily, 
    moredelim=**[is][{\btHL[facultatif]}]{@}{@},
    escapechar={§},
    aboveskip=-0.5ex,
    belowskip=-1ex,
    xleftmargin=-.25in,
    commentstyle=\color{black},
    stringstyle=\color{black}}
    
    
    % -----------------------------------------------------------------------
    
    \begin{document}
    
    \raggedright
    \footnotesize
    \begin{multicols*}{3}
    
    
    % multicol parameters
    % These lengths are set only within the two main columns
    %\setlength{\columnseprule}{0.25pt}
    \setlength{\premulticols}{1pt}
    \setlength{\postmulticols}{1pt}
    \setlength{\multicolsep}{1pt}
    \setlength{\columnsep}{2pt}
    
    \begin{center}
            \Large{\textbf{INF135 Aide-mémoire}} \\
    \end{center}
    
    \section{Types de données}
    \begin{tabular}{@{}ll@{}}
        double & Valeur numérique\\
        char & Valeur textuelle \\
        logical & Valeur Booléenne\\
        struct & Enregistrement
    \end{tabular}
    
    \section{Déclaration de variables et constantes}
    \begin{tabular}{@{}ll@{}}
        Constante & NB\_TERMES = 45;\\
        Variable & x = 5; \\
    \end{tabular}
    
    \section{Entrée et sortie dans la fenêtre de commande}
    \subsection{Entrée}
    ¬x = input('Entrez un nombre : ');¬
    ¬str = input('Entrez votre nom :', 's');¬
    \subsection{Sorties}
    ¬fprintf(fmt, A1, .., An)¬
    ¬fprintf('La valeur est: %g',nombre);¬
    \subsection{Format d'affichage pour fprintf et fscanf }
    \begin{tabular}{@{}ll@{}}
        Valeur numérique & \%g \\
        Valeur textuelle & \%s \\
        Précision décimale & \%.4f
    \end{tabular}
    \subsection{Caractère spéciaux pour fprintf}
    \begin{tabular}{@{}ll@{}}
        Saut de ligne & ¬\n¬ \\
        Tabulation & ¬\t¬ \\
        Apostrophe & ¬''¬ \\
        Pourcentage & ¬%%¬ \\
        Backslash & ¬\\¬
    \end{tabular}
    
    \section{Opérateurs}
    \begin{tabular}{@{}ll@{}}
    arithmétiques: & + - * / \^{}  mod(a,b)\\
    relationnels: & < <= == >= > \char`~=\\
    logiques: & \&\& || \char`~
    \end{tabular}
    
    \section{Conditionnelles}
    \begin{lstlisting}[mathescape]
    if §\mlplaceholder{condition}§
        §\mlplaceholder{instructions}§
    elseif §\mlplaceholder{condition}§ $\tikzmark{list-falc-in}$
        §\mlplaceholder{instructions}§
    else
        §\mlplaceholder{instructions}§ $\tikzmark{list-falc-out}$
    end
    %Exemple
    if x > 0
        fprintf('Positif')
    elseif x < 0
        fprintf('Négatif')
    else
        fprintf('Zéro')
    end
    \end{lstlisting}
    \AddNote{list-falc-in}{list-falc-out}{list-falc-in}{facultatif}
    
    
    \section{Boucles}
    \subsection{while}
    Quand on ne connait pas le nombre d'itérations
    \begin{lstlisting}
    while §\mlplaceholder{condition}§ 
        §\mlplaceholder{instructions}§
    end
    %Exemple
    i = 1;
    while i < 5
        fprintf('%g\n', i)
        i = i + 1;
    end
    \end{lstlisting}
    \subsection{for}
    Quand on connait le nombre d'itérations
    \begin{lstlisting}
    for i = a:@step:@b
        §\mlplaceholder{instructions}§
    end
    %Exemple
    for i = 1:10:100
        fprintf('%g', i)
    end
    \end{lstlisting}
    \begin{minipage}{1.0\textwidth}
    
        \section{Fonctions}
        \subsection{Définitions}
        \begin{lstlisting}[xleftmargin=\dimexpr-\leftmarginii-\leftmargini]
            function @[out1, out2] =@ nom(@in1, in2@)
            §\mlplaceholder{instructions}§
            out1 = §\mlplaceholder{expression}§
            out2 = §\mlplaceholder{expression}§
            end
        \end{lstlisting}
    \end{minipage}
        
        \subsection{Fonctions utiles}
    \begin{tabular}{@{}ll@{}}
    ¬clc¬ & Vide la fenêtre de commande \\
    ¬clear¬ & Vide la mémoire \\
    ¬fix(X)¬ & Coupe la partie fractionnaire \\
    ¬mod(X,M)¬ & Restant d'une division \\
    ¬sqrt(X)¬ & Racine carré \\
    ¬abs(X)¬ & Valeur absolue \\
    ¬round(X)¬ & Arrondie \\
    ¬randi(X)¬ & Nomre aléatoire entre 1 et X
    \end{tabular}

    \section{Validation}
    \subsection{Fonctions is* valide...}
    \begin{tabular}{@{}ll@{}}
    ¬isempty(X)¬ & Variable vide \\
    ¬isvector(X)¬ & Tableau 1D \\
    ¬ismatrix(X)¬ & Tableau 2D \\
    ¬isequal(a,b)¬ & Données égales \\
    ¬islogical(X)¬ & Valeur booléenne \\
    ¬isnumeric(X)¬ & Valeur numérique \\
    ¬isinf(X)¬ & Valeur infinie \\
    ¬isnan(X)¬ & Valeur impossible \\
    ¬ischar(X)¬ & Chaine de caractères \\
    ¬isstruct(X)¬ & Enregistrement \\
    ¬isa(X,classStr)¬ & Un objet d'une classe\\
    ¬isprime(X)¬ & Nombre premier\\
    ¬isfield(S,field)¬ & Nom d'un champ dans une struct\\
    \end{tabular}
    \subsection{Fonctions}
    \begin{tabular}{@{}ll@{}}
    ¬nargin¬ & Nombre de paramètres reçus\\
    ¬nargout¬ & Nombre de paramètre de sorties\\
    ¬error(msgStr)¬ & Génère une erreur \\
    ¬assert(cond)¬ & Génère une erreur si condition fausse
    \end{tabular}
    \subsection{validateattributes}
    ¬validateattributes(A, {type}, {attributes})¬
    \subsubsection{type}
    \begin{tabular}{@{}ll@{}}
    ¬'numeric'¬ & Type numérique (double)\\
    ¬'logical'¬ & Type logique (logical)\\
    ¬'char'¬ & Type textuelle (char)\\
    \end{tabular}
    \subsubsection{attributes}
    \begin{tabular}{@{}ll@{}}
    ¬'vector'¬ & Taille d'une vecteur \\
    ¬'scalar'¬ & Valeur unique 1x1\\
    ¬'2d'¬ / ¬'3d'¬  & Taille 2D ou 3D\\
    ¬'scalartext'¬ & Vecteur de caractères\\
    ¬'>' , N¬ & Valeur plus grande que N\\
    ¬'positive'¬ & Valeurs positives
    \end{tabular}

    \section{Tableau}
    \subsection{Fonctions de création}
    \begin{tabular}{@{}ll@{}}
    ¬[1, 2, 3]¬ & Vecteur manuel\\
    ¬[1, 2; 3, 4]¬ & Matrice manuelle\\
    ¬1:4¬ & Série numérique\\
    ¬zeros(n,m))¬ & n x m de zéros\\
    ¬ones(n,m))¬ & n x m de uns\\
    ¬rand(n,m)¬ & n x m aléatoire [0,1]\\
    ¬randi(imax,n,m)¬ & n x m aléatoire entre 1 et imax\\
    eye(n,m) & n x m matrice identité\\
    diag(v) & Matrice avec v diagonale
    \end{tabular}
    \subsection{Accès}
    \begin{tabular}{@{}ll@{}}
    ¬vec(i)¬ & Accès au vecteur à l'indice i\\
    ¬mat(i,j)¬ & Accès à la matrice à la ligne i et colonne j
    \end{tabular}
    \subsection{Fonctions}
    \begin{tabular}{@{}ll@{}}
    ¬numel(A)¬ & Nombre d'éléments du tableau\\
    ¬size(A,1)¬ & Nombre de ligne du tableau\\
    ¬size(A,2)¬ & Nombre de colonne du tableau\\
    ¬max(A)¬ & Maximum \\
    ¬min(A)¬ & Minimum\\
    ¬sum(A)¬ & Somme\\
    ¬mean(A)¬ & Moyenne\\
    ¬sort(A)¬ & Tri
    \end{tabular}
    \subsection{Fonctions sur les chaines de caractères}
    \begin{tabular}{@{}ll@{}}
    ¬strfind(str,pattern)¬ & Trouves les indices du pattern\\
    ¬replace(str,old,new)¬ & Remplace dans un chaine\\
    ¬[tok,rem] = strtok(str)¬ & Coupe la chaine aux espaces\\
    ¬strcmp(s1,s2)¬ & Compare deux chaines\\
    ¬strcmpi(s1,s2)¬ & Compare chaines, ignore casse\\
    ¬upper(str)¬ & Change la chaine en haut de casse\\
    ¬lower(str)¬ & Change la chaine en bas de casse\\
    ¬str2num(str)¬ & Converti la chaine en nombre\\
    ¬num2str(num)¬ & Converti un nombre en chaine
    \end{tabular}
    
    \section{Fichiers}
    \subsection{Fonctions générale}
    \begin{tabular}{@{}ll@{}}
    ¬fileID = fopen(filename)¬ & Ouvre en lecture\\
    ¬fileID = fopen(filename,'w')¬ & Ouvre en écriture\\
    ¬fclose(fileID)¬ & Ferme le fichier\\
    ¬fclose('all')¬ & Ferme tous les fichiers\\
    ¬feof(fileID)¬ & Détecte la fin du fichier
    \end{tabular}
    \subsection{Fichier texte}
    \begin{tabular}{@{}ll@{}}
    ¬ligne = fgetl(fileID)¬ & Lire une ligne de texte\\
    ¬ligne = fgets(fileID)¬ & fgetl avec ¬\n¬ inclus \\
    ¬A = fscanf(fid,fmt)¬ & Lire données formattées\\
    ¬fprintf(fileID,fmt,A1)¬ & Écrire dans fichier
    \end{tabular}
    \subsection{Fichier Binaire}
    \begin{tabular}{@{}ll@{}}
    ¬A = fread(fileID,sizeA,type)¬ & Lire dans fichier\\
    ¬fwrite(fileID,A,type)¬ & Écrire dans fichier
    \end{tabular}
    
    Type possible :
    \begin{tabular}{@{}ll@{}}
    ¬'int32'¬ & Nombre entiers\\
    ¬'double'¬ & Valeurs numérique\\
    ¬'char'¬ & Chaine de caractères
    \end{tabular}
    
    
    %\begin{minipage}{\linewidth}
    \section{Classe}
    \subsection{Définition}
    \begin{lstlisting}
    classdef NomClasse < handle
        properties
            p1
        end
        properties (SetAccess=private)
            p2
        end
        properties (Dependant)
            p3
        end
        methods
            function [obj] = NomClasse(in1,in2)
                obj.p1 = in1 + in2;
            end
            function [prop3] = get.p3(obj)
                prop3 = obj.p1 + obj.p2;
            end
            function [out] = method1(obj,inArg)
                out = obj.p1 + inArg;
            end
            function method2(obj,inArg)
                obj.p2 = obj.p1 + inArg;
            end
        end
    end
    \end{lstlisting}
    \vspace{0.2in}
%   \begin{minipage}{1.0\textwidth}
    \subsection{Utilisation d'objets}
    \begin{lstlisting}
    %Création d'objet
    obj = NomClasse(1,2);
    %Accès aux propriétés publiques
    obj.p1 = 3;
    %Appel des méthodes
    x = obj.method1(4)
    obj.method2(6)
    \end{lstlisting}
%\end{minipage}
    %\end{minipage}
    \subsection{Attributs de propriétés}
    \begin{lstlisting}
    % Lecture Seule
    SetAccess = private
    % Écriture Seule
    GetAccess = private
    % Accès privées
    Access = private
    % Constante
    Constant = true
    % Dépendante
    Dependent
    \end{lstlisting}
    
    \end{multicols*}
    \end{document}
    
    
    
    